\documentclass[11pt, a4paper]{../awesome-cv}
\usepackage{longtable}
\usepackage{svg}
\usepackage{fp} % Required for invoice calculations
\usepackage{advdate} % Required for date calculation
\usepackage[group-separator={,},group-minimum-digits=4, detect-all]{siunitx} % Required for automati
\usepackage{colortbl} % Required for colouring table cells (used for rules)
\usepackage{booktabs} % Required for nicer table rules
\usepackage{multirow} % Required for allowing cells to take up multiple rows in tables
\usepackage{makecell}
\usepackage{array} % Required for customizing table spacing and features
\def\arraystretch{1.2} % Table row spacing, 1 is the default
\geometry{left=1.4cm, top=.8cm, right=1.4cm, bottom=1.cm, footskip=.2cm}
\fontdir[../fonts/]
\colorlet{awesome}{awesome-darknight}
\setbool{acvSectionColorHighlight}{false}

\renewcommand{\acvHeaderSocialSep}{\quad\textbar\quad}
\hypersetup{%
  pdftitle={J. P. Lucero Lorca - Invoice},
  pdfauthor={Jose Pablo Lucero Lorca},
  pdfsubject={Invoice},
  pdfkeywords={applied mathematics, numerical analysis, nuclear engineering},
  colorlinks=false,% hyperlinks will be black
  urlbordercolor=gray,% hyperlink borders will be red
  pdfborderstyle={/S/U/W 0.2}% border style will be underline of width 1pt
}

\name{José Pablo}{Lucero Lorca}
\position{applied mathematician - numerical analyst - nuclear
  engineer}
\address{3080 Pearl Parkway, APT D-415, Boulder, CO, 80301}
\mobile{+1(720)919-9118}
\email{jpablo.lucero@gmail.com}
\homepage{https://pablo.world/math}
\github{jpablolucero}
\linkedin{jpablolucero}
\recipient
{EMA}
{\hspace{0cm}}

\definecolor{highlightcolour}{HTML}{000000} % Colour used for making text stand out
\definecolor{rulecolour}{HTML}{000000} % Colour used for rules

\newcommand{\payeename}[1]{\renewcommand{\payeename}{#1}}
\newcommand{\payeejob}[1]{\renewcommand{\payeejob}{#1}}
\newcommand{\payeeaddresslineone}[1]{\renewcommand{\payeeaddresslineone}{#1}}
\newcommand{\payeeaddresslinetwo}[1]{\renewcommand{\payeeaddresslinetwo}{#1}}
\newcommand{\payeecontactlineone}[1]{\renewcommand{\payeecontactlineone}{#1}}
\newcommand{\payeecontactlinetwo}[1]{\renewcommand{\payeecontactlinetwo}{#1}}

\newcommand{\invoiceref}[1]{\renewcommand{\invoiceref}{#1}}
\newcommand{\invoiceissued}[1]{\renewcommand{\invoiceissued}{#1}}
\newcommand{\invoicedue}[1]{\renewcommand{\invoicedue}{#1}}
\newcommand{\projectname}[1]{\renewcommand{\projectname}{#1}}

\newcommand{\companyname}[1]{\renewcommand{\companyname}{#1}}
\newcommand{\sendername}[1]{\renewcommand{\sendername}{#1}}
\newcommand{\senderjob}[1]{\renewcommand{\senderjob}{#1}}
\newcommand{\senderaddresslineone}[1]{\renewcommand{\senderaddresslineone}{#1}}
\newcommand{\senderaddresslinetwo}[1]{\renewcommand{\senderaddresslinetwo}{#1}}
\newcommand{\sendercontactlineone}[1]{\renewcommand{\sendercontactlineone}{#1}}
\newcommand{\sendercontactlinetwo}[1]{\renewcommand{\sendercontactlinetwo}{#1}}

\newcommand{\termsandconditions}[1]{\renewcommand{\termsandconditions}{#1}}

\newcommand{\taxrate}[1]{\renewcommand{\taxrate}{#1}} % Tax rate used to automatically calculate tax

% Cumulative variables
\gdef\TotalA{0} % Total before tax
\gdef\TotalB{0} % Tax
\gdef\TotalC{0} % Net after tax
\taxrate{0.0} % The tax rate used to automatically calculate tax, must be set to a decimal between 0 and 1 (for no tax set to 0)

%------------------------------------------------
% Invoice information

\invoiceref{AAA202410200} % Invoice number
\invoiceissued{\today} % Date issued
\invoicedue{\DayAfter[10]} % Due date, a number of days into the future, use 0 for due on receipt
\projectname{Charge Plus Optimization}

%------------------------------------------------
% Recipient/payee information

\payeename{Electro Magnetic Applications Inc.}
\payeejob{}
\payeeaddresslineone{143 Union Blvd \#900, Lakewood, CO, 80228, United States of America}
\payeeaddresslinetwo{}
\payeecontactlineone{Timothy McDonald tim@ema3d.com}
\payeecontactlinetwo{}

% ------------------------------------------------

\newcommand{\invoiceitem}[3]{

  % Calculation of running variables (current line)
  \FPmul\gross{#2}{#3}\FPround\gross{\gross}{2} % Calculate the current gross (quantity * rate)
  \global\let\variableA\gross % Set variable for use in table and other calculations
  
  % ------------------------------------------------
  
  % Calculation of cumulative variables (total for invoice)
  \FPeval{\beforetax}{round(\TotalA + \variableA, 2)} % Add the current before tax value to the previous total
  \global\let\TotalA\beforetax % Set variable for display at the end and in other calculations
  
  \FPeval{\tax}{round(\TotalA * \taxrate, 2)} % Calculate the updated total tax to pay
  \global\let\TotalB\tax % Set variable for display at the end and in other calculations
  
  \FPeval{\aftertax}{round(\TotalA + \TotalB, 2)} % Calculate the updated total amount due
  \global\let\TotalC\aftertax % Set variable for display at the end
  
  % ------------------------------------------------
  
  \vspace{-0.35cm}
  % Output the table row
  #1 & % Item name
  \num{#2} & % Rate
  \num{#3} & % Rate
  \$\num{\variableA} \\\cline{1-1}  % Subtotal
}

\newenvironment{invoicetable}
{
  \begin{longtable}{p{0.6\textwidth} R{0.1\textwidth}  R{0.1\textwidth}        R{0.1\textwidth}}\hline
                   \textbf{ITEM}  & \textbf{TIME}     & \textbf{RATE}        & \textbf{SUBTOTAL} \\
                                  & \textbf{[min]}    & \textbf{[\$/minute}] & \\
    \arrayrulecolor{rulecolour}\toprule[0.5pt]
    }{
%    \arrayrulecolor{rulecolour}\bottomrule[0.5pt]
    \\ % Reduce whitespace before the totals
                                  && Total & \$\num{\TotalA} \\
  \end{longtable}
}

\begin{document}

% Print the header with above personal informations
% Give optional argument to change alignment(C: center, L: left, R: right)
\makecvheader[R]

% Print the footer with 3 arguments(<left>, <center>, <right>)
% Leave any of these blank if they are not needed
\makecvfooter
  {\today}
  {José Pablo Lucero Lorca~~~.~~~Invoice}
  {}

  \footnotesize
  \begin{tabular}{p{0.25\textwidth} p{0.22\textwidth} p{0.22\textwidth} p{0.22\textwidth}}
    \arrayrulecolor{rulecolour}\toprule[0.5pt] % Horizontal line at the top of the table
    \multirow{2}{*}{{\color{highlightcolour} \Huge INVOICE}} & \textbf{Recipient} & \\
                                                             & \payeename & \payeeaddresslineone & \payeecontactlineone \\
    \arrayrulecolor{rulecolour}\bottomrule[0.5pt] % Horizontal line at the bottom of the table
  \end{tabular}

  \vspace{-0.2cm}
  \footnotesize
  \begin{tabular}{p{0.25\textwidth} p{0.66\textwidth}}
    {\bf Electronic payment information} & Wells Fargo - Account number 8094382838 - Routing number 102000076
  \end{tabular}
  \vspace{-0.2cm}

  \footnotesize
  \begin{tabular}{p{0.225\textwidth} p{0.225\textwidth} p{0.225\textwidth} p{0.225\textwidth}}
    \arrayrulecolor{rulecolour}\toprule[0.5pt] % Horizontal line
    \textbf{Invoice Number} & \textbf{Date} & \textbf{Payment Due} & \textbf{Project Name} \\
    \arrayrulecolor{rulecolour}\toprule[0.5pt] % Horizontal line
    \invoiceref & \invoiceissued & \invoicedue & \projectname \\
  \end{tabular}
  \vspace{-0.3cm}
  
  \begin{invoicetable}
    \bf{September} & & & \\\cline{1-1}
    \invoiceitem{
      \texttt{\bf EMA3D-1418} \vspace{0.1cm}\\
      
      Transfer new test case files from Koi with
      modifications. Include test case in test suite and run to check that
      no \texttt{NaN}’s are present. The case run successfully in 50min. Transfer
      output back in case Bryon wants to take a look.
    }{10}{5}
    \invoiceitem{
      \texttt{\bf EMA3D-1418} \vspace{0.1cm}\\

      Instrumentation of \texttt{PIC::UpdateNew()} separated in meaningful
      sections. Will report relevant line sections and time percentages. Had
      to deal with an instability on Linux where the code stalls, will
      report it back to Bryon and continue to try to find the key line
      producing it.
    }{120}{5}
    \invoiceitem{
      \texttt{\bf EMA3D-1418} \vspace{0.1cm}\\

      Instability associated to \texttt{Rate Preprocess = True} identified and
      reported in comment.  Tip of \texttt{branch1} has been modified in such a way
      that the instability no longer appears, so working with new commit
      until further notice about the issue.
    }{60}{5}
    \invoiceitem{
      \texttt{\bf EMA3D-1418} \vspace{0.1cm}\\
      
      Identification of repeated calls to \texttt{vector <double>
        efield = RN.fem\_->EFE(el\_num,xval)} as the source of {\bf 40\%} of
      the time in \texttt{PIC::UpdateNew(double)} investigating algorithmic
      improvements for reading \texttt{efield}.
    }{30}{5}
    \bf{October} & & & \\\cline{1-1}
    \invoiceitem{
      \texttt{\bf EMA3D-1418} \vspace{0.1cm}\\

      Identification of unnecessary code in
      \texttt{TE4SubStruct::GradValues}, plus improvements in constness and
      redirection led to a {\bf ~70\%} speedup in the call to
      \texttt{RN.fem\_->EFE} inside \texttt{PIC::UpdateNew(double)}, and overall {\bf
        $\sim$20\%} in \texttt{PIC::UpdateNew(double)}.
    }{90}{5}
    \invoiceitem{
      \texttt{\bf EMA3D-1418} \vspace{0.1cm}\\

      Bring the code actually being executed to obtain \texttt{efield}
      for the \texttt{RF\_cube} case provided by Bryon out of the virtual
      table lookup. Result is {\bf 330\%} speedup to obtain \texttt{efield}
      and {\bf 42\%} speedup overall in \texttt{PIC::UpdateNew(double)}
      Report to Bryon through JIRA.
    }{180}{5}
    \invoiceitem{
      \texttt{\bf EMA3D-1418} \vspace{0.1cm}\\

      New measurements indicate the {\bf $\boldsymbol{\sim}$2 million calls per timestep} to
      \texttt{int ReadNew::PIC::CurrentElement(vector <double>)}, now take
      {\bf 20\%} of the time in void
      \texttt{ReadNew::PIC::UpdateNew(double)}. Looking into the necessity of the
      costly search performed in that function.
    }{60}{5}
    \invoiceitem{
      \texttt{\bf EMA3D-1418} \vspace{0.1cm}\\

      New measurements on Bryon’s last commit as of today on Linux, confirming
      a {\bf 200\%} speedup of \texttt{ReadNew::PIC::UpdateNew(double)} after
      \texttt{efield} optimization. Reporting back
      to Bryon with answer to inquiry about space filling curves with
      bibliography.
    }{30}{5}
    \invoiceitem{
      \texttt{\bf EMA3D-1418} \vspace{0.1cm}\\

      Analyzed \texttt{int ReadNew::PIC::CurrentElement(vector
        <double>, vector <double>, Uint)}.  Found
      out that it spends most of the time querying a \texttt{vector <double>
        kElemMap} and not a lot can be gained modifying the code as it stands
      for now. Shifted focus to other parts of the code.
    }{60}{5}
    \invoiceitem{
      \texttt{\bf EMA3D-1418} \vspace{0.1cm}\\
      \nopagebreak
      Simplifying the calculation of \texttt{energy\_1},
      \texttt{energy\_2} and \texttt{energy\_3} and taking away branching for
      vectorization led to {\bf 35\%} speedup measured. Informed Bryon
      on comment.
    }{120}{5}
    \invoiceitem{
      \texttt{\bf EMA3D-1418} \vspace{0.1cm}\\

      New timings breakdown, identification of new main objectives of
      optimization, report to Bryon through JIRA.
    }{60}{5}
    % \invoiceitem{
    %   \texttt{\bf EMA3D-1418} \vspace{0.1cm}\\
    % }{}{5}
    % \invoiceitem{
    %   \texttt{\bf EMA3D-1418} \vspace{0.1cm}\\
    % }{}{5}
    % \invoiceitem{
    %   \texttt{\bf EMA3D-1418} \vspace{0.1cm}\\
    % }{}{5}
  \end{invoicetable}


\end{document}
